
\subsection{Scholarly Work}
Scholarly research in transportation model distinguishes between two
broad approaches: \cite{kitamura1988}

\begin{description}[style=nextline]
    \item[Activity-based] Modeled at the individual level. Considers
  trips arising from different activities that comprise a tour. (E.g.,
  travel to kids' school, then to work, then to the grocery store, and
  finally back home). \cite{kitamura1988}
    \item[Trip-based] Modeled at the Traffic Analysis Zone (TAZ)
  level. Considers broader characteristics such as demographics of a
  neighborhood to model trip times and volumes.\cite{murthy01}
\end{description}

The activity-based model dominates in the most recent academic
literature, but from our survey, the trip-based model still appears to
be the most frequently used type of model by government
agencies. Within the trip-based models the dominant kind appears to be
the four-stage model, which breaks down into four components:
\cite{murthy01}

\begin{enumerate}
    \item Trip Generation
    \item Trip Distribution
    \item Mode Choice
    \item Trip Assignment
\end{enumerate}

In all the applications saw, each of those stages is further refined
by the category of the trip being modeled. The usual breakdown is:

\begin{description}
  \item[Home-based work] (HBW) Trips from home to work or vice versa.
  \item[Home-based shop] (HBS) Trips from home to go shopping and back
(sometimes omitted).
  \item[Home-based other](HBO) Other trips having the home as an
origin or destination.
  \item[Non-home-based work] (NHW) Trips to or from work not in HBWl
  \item[Non-home-based other](NHO) All other kinds of trips.
\end{description}

One of the main benefits of the trip-based approach is its long
history and current widespread use. The model is also readily
modularized, and was easily adapted and specialized for our
purposes. On the other hand, it is more simplistic than the
activity-based model, since it ignores the interactions of past trips
generated on future trip generation. The activity-based model,
however, has a clear computational disadvantage. The level of detail
the model reaches means that it is very demanding computationally, and
is also hard to break into subcomponents.

Our approach combined aspects from both styles (as detailed in section
3), but borrowing the terminology of the trip-based style.

\subsubsection{Phoenix, Arizona Planning Authority}
In 2011 the planning authority in Phoenix, Arizona, MAG, undertook an
extensive project to model transit movement to planned special events
in the region. \cite{kuppam11} Our chosen topic, sports games, falls
under their category of a planned special event. The authors identify
the proportion of special events patrons who utilized light rail
v. alternative modes of transportation and their approach to modeling
demand and modal choice for this event. While the results are specific
to their application, the methodology helped inform our own approach.
focus.

\subsubsection{Robertson Stadium}
Gunda Corporation performed a relatively similar project to the one we
developed for Robertson Stadium at the University of
Houston.\cite{gunda} Although the stadium has a lower capacity than
the Phillies stadium, \cite{robertson-stadium} it is home to an MLS
team, and has to abide by all the traffic management principles of a
professional sports team. The scope and aim of Gunda's analysis
differed from our own, but their analysis helped guide our design of
a component of our model. (The micro-car module, see section 3)

\subsection{Time-Expanded Graphs}
Much of our work on the SEPTA model, discussed in section 3, was
informed by the work of Frank Schulz in his
dissertation\cite{schulz2005timetable}. The timetable routing problem
is a special class of traffic assignment problem in which time is
naturally discretized (by the timetable). Because of this special
structure, these problems don't need to rely on the inaccurate or
computationally expensive algorithms that discretize time. Instead,
the \emph{time-expanded graph} refers to a graph where nodes
correspond to moments in time at a particular location (e.g., Grand
Central at 3:15pm). Edges correspond to direct transit lines between
two space-time locations.

Although we did not venture into the more sophisticated algorithms
described in \cite{kohler2002time}, the discussion about their
computational efficiency helped us make certain design choices
discussed in section 3. Also of use in this regard were
\cite{xuan2003computing} and \cite{george2008time}.


\subsection{Queue-Based Modeling}
Our work on the micro car model, also discussed in section 3, took a
completely different approach to that of the SEPTA model. The
queue-based approach that we used is described in great detail in
\cite{van2007modeling}. As described by Van Woensel and Vandaele,
queueing models are primarily used to model individual intersections,
as in \cite{ruskin2002modeling} or \cite{kyte2009validating}. These
queueing models operate by considering car arrivals as a stochastic
process with a single server at the intersection. The focus of these
types of models is more on the interaction of multiple cars at an
intersection, like in \cite{prasetijo2012capacity}.

Our model differed in two important ways from those mentioned above

\begin{enumerate}
    \item All the cars in the stadium parking lot are headed in the
  same direction: \emph{out}
    \item The parking lot we considered has over 100 nodes
\end{enumerate}

\subsection{Dynamic Traffic Assignment}
When specialized to cars, the assignment problem can be solved in a
variety of ways. One particularly high-level distinction is between
static and dynamic assignment, which can be thought of roughly as the
difference between the steady-state and the transient solution to the
problem. The algorithm we developed had at its core the ideas
developed by Dial \cite{dial2006path} and Nie
\cite{nie2010class}. They describe a class of algorithms to
efficiently solve the dynamic traffic assignment problem and a metric
called user equilibrium (UE) to measure the effectiveness of these
algorithms. Boyce \cite{boyce2003convergence} has an extended
discussion of user equilibrium.

\subsection{Group Members' Prior Work}
One of our group members had had experience working with Penn Transit
to improve the dispatching efficiency of their bus operations around
campus. This work allowed for in-depth understanding of queuing theory
and demand generation, which was extremely useful in our micro car
module, described in more detail in section 3. Using elements of the
prior project, combined with additional concepts from traffic
engineering, we developed an accurate model.
