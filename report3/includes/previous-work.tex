
\subsection{Scholarly Work}
Scholarly research in transportation model distinguishes between two
broad approaches: \cite{kitamura1988}

\begin{description}[style=nextline]
    \item[Activity-based] Modeled at the individual level. Considers
  trips arising from different activities that comprise a tour. (E.g.,
  travel to kids' school, then to work, then to the grocery store, and
  finally back home). \cite{kitamura1988}
    \item[Trip-based] Modeled at the Traffic Analysis Zone (TAZ)
  level. Considers broader characteristics such as demographics of a
  neighborhood to model trip times and volumes.\cite{murthy01}
\end{description}

The activity-based model dominates in the most recent academic
literature, but from our survey, the trip-based model still appears to
be the most frequently used type of model by government
agencies. Within the trip-based models the dominant kind appears to be
the four-stage model, which breaks down into four components:
\cite{murthy01}

\begin{enumerate}
    \item Trip Generation
    \item Trip Distribution
    \item Mode Choice
    \item Trip Assignment
\end{enumerate}

In all the applications we've seen, each of those stages is further
refined by the category of the trip being modeled. The usual breakdown
is:

\begin{description}
  \item[Home-based work] (HBW) Trips from home to work or vice versa.
  \item[Home-based shop] (HBS) Trips from home to go shopping and back
(sometimes omitted).
  \item[Home-based other](HBO) Other trips having the home as an
origin or destination.
  \item[Non-home-based work] (NHW) Trips to or from work not in HBWl
  \item[Non-home-based other](NHO) All other kinds of trips.
\end{description}

One of the main benefits of the trip-based approach is its long
history and current widespread use. The model is also readily
modularized, and can be easily adapted and specialized for our
purposes. On the other hand, it is more simplistic than the
activity-based model, since it ignores the interactions of past trips
generated on future trip generation. The activity-based model,
however, has a clear computational disadvantage. The level of detail
the model reaches means that it is very demanding computationally, and
is also hard to break into subcomponents.

Our approach will combine aspects from both styles (as detailed in
section 3), but will borrow the terminology of the trip-based style.

\subsubsection{Phoenix, Arizona Planning Authority}
In 2011 the planning authority in Phoenix, Arizona, MAG, undertook an
extensive project to model transit movement to planned special events
in the region. \cite{kuppam11} Our chosen topic, sports games, falls
under their category of a planned special event. The authors identify
the proportion of special events patrons who utilized light rail
v. alternative modes of transportation and their approach to modeling
demand and modal choice for this event. While the results are specific
to their application, the methodology can be adapted to our specific
focus.

\subsubsection{ITE Trip Generation Database}
The ITE compiled a database with various data patterns related to Trip
Generation. \cite{ite08} We will be able to use the information it contains,
specifically the information related to the special events travel,
e.g. sports games and movies.

\subsubsection{Robertson Stadium}
Gunda Corporation performed a relatively similar project to the one we
are developing for Robertson Stadium at the University of
Houston.\cite{gunda} Although the stadium has a lower capacity than
the Eagles stadium, \cite{robertson-stadium} it is home to an MLS
team, and has to abide by all the traffic management principles of a
professional sports team. The scope and aim of Gunda's analysis
differed from our own, but their analysis can strongly inform a
component of our model. (The micro-car module, see Section \ref{cars})

\subsection{Group Members' Prior Work}
One of our group members has had experience working with Penn Transit
to improve the dispatching efficiency of their bus operations around
campus. This work has allowed for in-depth understanding of queuing
theory and demand generation, which will be extremely useful in our
module about fans taking cars to the game. Using elements of the prior
project, combined with additional concepts from traffic engineering,
we will be able to develop an accurate model.
