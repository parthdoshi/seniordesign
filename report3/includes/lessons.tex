t is important to consistently review and assess your work in order to derive the maximum benefit from any project and in order to improve the current and future projects. When conducting a review, one should not only focus on the mistakes, but also examine what was done well in order to ensure that it is followed again. Furthermore, one should analyze the mistakes and understand the underlying reasons that lead to those mistakes so that they are not made again. Throughout our project we had review meetings, however at the end it is most effective when you can look back at the end project and analyze the project as a whole. As a result of our analysis, below are the main lessons we learned:

\begin{itemize}
  \item The most important lesson learned was with respect to scheduling and particularly the importance of including slack time and regularly updating your schedule. We put a lot of effort into creating and updating the schedule in order to clearly know how much work was left to be done and by what date it had to be completed. By not including slack time, when we had setbacks, such as not being able to set up a meeting with the Phillies, we fell behind schedule and had to do a disproportionately large amount of work towards the end. However, we would change the order of tasks to incorporate the delays and because we updated our schedule regularly we were always aware of our current situation.

  \item It is very important to set deadlines for important decisions that can greatly impact your project. We had not set a deadline to decide who our end user was because we kept waiting on the Phillies to meet us. This decision impacted the model and particularly the GUI. If we had set at deadline and chosen earlier, we would have had more time to improve our GUI (website) for the fans and possibly even been able to launch it.

  \item There is always going to be a tradeoff between runtime and accuracy in any simulation and it is important to decide on what is the right balance based on your user requirements. This is an important decision to make early on so that you can code your model accordingly and not have to adjust your model later on.

  \item One thing that helped us greatly was having system block diagrams with inputs and outputs of each block clearly specified. This enabled us to create each subsystem knowing what inputs were available and what outputs had to be produced. This made the integration of the system very easy.

  \item Good software documentation and choice of software can make your project much easier. Firstly, one software should be chosen at the start of the project that everyone is comfortable with because when coding is done on multiple software it becomes difficult to integrate. Secondly, all code should be excessively documented in order to enable team members to work on it together and it also makes debugging easier.

  \item It is important and can save you a lot of time if you do a thorough search of previous work before you start your project. This is because if you find it late you will have redone the work and wasted a lot of resources.

\end{itemize}
