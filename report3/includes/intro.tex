\section{Introduction}
\subsection{Overview}
Stadium Traffic aims to reduce the greenhouse gas (GHG) emissions from
people going to and from sports games. In particular, our project
focuses on the Citizens Bank Park where fans of the Philadelphia
Phillies support their team at home games. We have built a model that
can be used to quantify the GHG emissions generated due to people
coming to and going from games using cars as well as public
transportation. We have created a front-end website that fans can use
to understand the environmental impact of their mode-choice in
traveling to and from home games. Our project is endorsed by the
Philadelphia Phillies. They have recognized our efforts toward
reducing GHG emissions due to fans traveling to and from these home
games and reducing congestion in the parking lots at the stadium by
encouraging fans to switch to SEPTA.

\subsection{Motivation}
All of us have been to large gatherings like sports games, concerts or
conferences and can therefore relate to the traffic problems when
everyone is trying to leave at the same time at the end of the
event. However, what most people don't realize is the environmental
effect of this traffic. Our project aims to find solutions to reduce
this traffic and hence reduce the emissions while cars idle in parking
lots trying to leave.

The project is focusing on the Philadelphia Phillies stadium, but the
underlying models adaptable to other venues hosting large-scale
events. Sports events draw huge crowds at peak traffic hours and are
therefore a great instance on which to model our project. The Phillies
are also undertaking the \'Red Goes Green\' initiative, which is a push
towards becoming more environmentally conscious. Some initiatives that
they have taken are the use of solar energy sources for their power
consumption and partnering with the Pennsylvania Horticultural Society
to plant one million trees to counteract the aforementioned carbon
emissions. We believe that our project will be helpful to the Phillies
to integrate into their Red Goes Green initiative and further reduce
the carbon footprint of the stadium.

When a typical baseball game ends, everyone tries to the leave the
stadium at the same time. Although traffic police conduct the traffic,
they do it in a haphazard and unscientific manner. Furthermore, the
decision of which gates to be open and which ones to remain closed is
also done using intuition rather than any efficiency-maximizing
algorithm. We saw this inefficient process as an opportunity to
develop an algorithm to reduce car idling time and GHG emissions.

\subsection{Project Goal and Objectives}
Although the overarching aim of our project is to reduce GHG emissions
due to people going to/from the sports games, we broke this down into
certain specific, measurable objectives:

\begin{enumerate}
    \item Develop a comprehensive model of transportation to and from
  the stadium.
    \item Quantify the total emissions due to transportation to and
  from games.
    \item Optimally route exiting vehicle traffic according to total
  emissions generated.
    \item Inform fans of the change in travel time and carbon
  emissions by using public transport rather than cars.
\end{enumerate}

The success of each of the objectives can be measured and how we would
measure the success is stated below:

We would consider a success model one that takes inputs such as trip
distribution, number of attendees, start and end time and produces
outputs such as average idling time and number of cars.

The aim of the project is to reduce GHG emissions, so we need to be
able to measure and quantify GHG emissions from idling cars.

Our model should be able to quantify the effect of different
incentives on GHG emissions so that we can test different strategies
and find the optimal combination.

Optimal routing of traffic can impact GHG emissions from idling cars,
so we need to create an algorithm that can take into account which
gates are open or closed and simulate appropriate car routing.

A front-end system that allows fans to understand the impact of their
decisions would allow our project to have the greatest impact. The
system should allow for alternatives should fans decide that they wish
to opt for the less environmentally-friendly option. A mock-up of a
website would be ideal to accomplish this goal.

\subsection{Constraints}
The main constraint is the cooperation of the Philadelphia Phillies
since they would act as our primary data source. Through the course of
the project, we had difficulties establishing concrete contact with
the Phillies and had to resort to alternative data collection and
measurement techniques. However, we have received the Phillies
endorsement for our project and hope that it will be officially
integrated with their Red Goes Green initiative in the near future.

Another constraint is the team's lack of experience with
transportation engineering. Since the project involves the
understanding, modeling and optimizing of transport systems, this
would be a constraint. However, we have taken an initiative to get the
help of Professor Vukan Vuchic, a senior professor in transport
engineering, and began familiarizing ourselves with relevant transport
systems concepts under his guidance.
