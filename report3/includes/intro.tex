\subsection{Overview}
Stadium Traffic aims to reduce the greenhouse gas (GHG) emissions from
people going to and from sports games. In particular, our project will
focus on the Philadelphia Eagles and Phillies stadiums.  We aim to
build a model that can be used to quantify the GHG emissions generated
due to people coming to and going from games. We will then develop
solutions to reduce the number of people coming by cars or leave over
longer time period and develop a traffic routing algorithm, aiming to
reduce the GHG emissions from idling cars waiting to leave the
stadium. We will test these on our model to find a successful
combination. If we can get the cooperation of Phillies/Eagles/Flyers
and Philadelphia Police Department, we hope to work with them to
implement the results of our project.

\subsection{Motivation}
All of us have been to large gatherings like sports games, concerts or
conferences and can therefore relate to the traffic problems when
everyone is trying to leave at the same time at the end of the
event. However, what most people don't realize is the environmental
effect of this traffic. Our project aims to find solutions to reduce
this traffic and hence reduce the emissions while cars idle in parking
lots trying to leave.

The project is focusing on the Philadelphia Eagles, Phillies and
Flyers stadiums. Sports events draw huge crowds at peak traffic hours
and are therefore a great instance on which to model our project. The
Eagles are also undertaking a Go Green! initiative, which is a push
towards becoming more environmentally conscious. Some initiatives that
they have taken are the use of renewable energy sources for their
power consumption and using cups and plates made from recycled
materials. We believe that we can get the support of the Eagles for
our project, as it can be part of the Go Green! initiative and further
reduce the carbon footprint of the stadium.

When the football or baseball game ends, everyone tries to the leave
the stadium at the same time. Although traffic police conduct the
traffic, they do it in a haphazard way. Furthermore, the decision of
which gates to be open and which ones to remain closed is also done
using intuition rather than any efficiency-maximizing algorithm. We
saw this inefficient process as an opportunity to develop an algorithm
to reduce car idling time and GHG emissions.

\subsection{Project Goal and Objectives}
Although the overarching aim of our project is to reduce GHG
emissions due to people going to/from the sports games, we broke this
down into certain specific and measurable objectives:

\begin{enumerate}
    \item Develop a comprehensive model of transportation to and from
  the stadium.
    \item Quantify the total emissions due to transportation to/from
  games.
    \item Develop a tool to simulate the emissions impact of various
  initiatives to reduce private transportation usage (e.g. offering a
  discounted drink to someone who rides the SEPTA).
    \item Create an algorithm to optimally route exiting vehicle traffic
  according to total emissions generated.
    \item Develop a method for the Philadelphia traffic police to
  implement the traffic routing recommendations of our project.
\end{enumerate}

The success of each of the objectives can be measured and how we would
measure the success is stated below:

We would consider a success model one that takes inputs such as trip
distribution, number of attendees, start and end time and produces
outputs such as average idling time and number of cars.

The aim of the project is to reduce GHG emissions, so we need to be
able to measure and quantify GHG emissions from idling cars.

Our model should be able to quantify the effect of different
incentives on GHG emissions so that we can test different
strategies and find the optimal combination.

Optimal routing of traffic can impact GHG emissions from idling
cars, so we need to create an algorithm that can be constraint
optimized. A working algorithm can help the traffic police conduct
traffic in a more organized manner.

Having a front-end system that can be used by the Philadelphia Police
will allow our algorithm to be implemented and hence allow our project
to have an impact. Therefore we need a mock-up of a computer or phone application
that can be used by the police while conducting traffic.

\subsection{Constraints}
The main constraint is the cooperation of the Philadelphia
Eagles/Phillies/Flyers and Philadelphia Police Department because without
their cooperation we cannot implement our project. However, our
advisor Dr. Huemmler has good relationships with both sets of organizations
and has been trying to get their support for the project.

Another constraint is the team's lack of experience with
transportation engineering.  Since the project involves the
understanding, modeling and optimizing of transport systems, this
would be a constraint. However, we have taken an initiative to get the
help of Professor Vukan Vuchic, a senior professor in transport
engineering, and began familiarizing ourselves with relevant transport
systems concepts under his guidance.

Finally, our last constraint is the data collection required to build
a representative model. However, we have been trying to get the
required information from the Eagles, Phillies, Flyers and other government
departments.

