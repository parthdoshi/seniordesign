

\subsection{System Specification}

We derive the specifications for our system based on the requirements
set by their running environments, especially those of the GUI. Since
we chose a website for this interface, the primary requirement was for
there to be a response time of under two seconds, since this is the
industry-standard benchmark \cite{akamai}, and what sites such as
Google use \cite{two-seconds}.

Response time for our website is driven by a variety of factors
typically broken down into:

\begin{itemize}
  \item The number of \emph{resources}
  \item The DNS lookup time
  \item The server response time
  \item The download time
  \item The rendering time
\end{itemize}

These times can be seen in a benchmark of an early version of our
website in figure \ref{fig-website-benchmark}.

\begin{figure}[htp]
  \centering
  \includegraphics[height=8cm]{graphics/performance.png}
  \caption{A `Waterfall View' of the stages of a website retrieval. In
  dark green \emph{DNS Lookup}; in orange \emph{Initial Connection};
  in light green \emph{Server processing time}; in blue \emph{Download
  Time}. The green bar shows the beginning of the rendering process,
and the blue line shows the end.}
  \label{fig-website-benchmark}
\end{figure}

The main parameters under our control for this interface were the
\emph{number of resources} and \emph{download time} and the \emph{server
  response time}. Based on some experimentation, we developed the
benchmarks for the first two items shown in table \ref{specs}. The
benchmark for \emph{server response time}, as discussed above, was the
main constraint in the design of this front end, so we set the
specification for this part as high as possible while remaining
withing the overall 2 second requirement, as shown in table
\ref{specs}.

To generate the response, the server must carry out 3 tasks:

\begin{itemize}
    \item Start-up script
    \item Process user input
    \item Form response HTML
\end{itemize}

By timing a variety of CGI scripts, and comparing the results to
published benchmarks\cite{cgi-benchmark} we determined an acceptable
performance range for the start-up time. The remaining time from the
original 2 seconds was then allocated jointly to processing user input
and forming the response HTML. The requirements are set out in table
\ref{specs}.

Another important dimension for the CGI script was its memory
footprint. Since the website was designed as a prototype, it would not
need to handle more than 3 concurrent requests. Given the limit of
30MB of memory in our our environment, we had to cap the CGI to under
10MB of memory, as shown in table \ref{specs}.

More qualitative requirements for the website regarded its aesthetic
appeal and functionality. The website had to handle improper inputs
robustly and be simple to use for a user of average computer literacy.
An important aspect of the robustness was that one user should not be
able to affect another user's experience with the website.

Given the design for our overall system, the driving constraints for
the back-end model were two-fold. On the one hand, the model had to be
fast enough to run at least once per game, which meant that, because
of the way baseball schedules work, the model had to run within one
day. On the other hand, the targeted user-base was initially small, so
the model had to be able to run on standard server hardware. We
specified this latter requirement to mean that the model would run on
a computer running Ubuntu 12.04 with a memory footprint under 3GB.

The modularity of our system enabled us to run different subsystems in
parallel, increasing memory use but reducing total processing time. We
could, however, run the models sequentially and thus limit the memor
consumption. The choice of execution strategy would dictate which of
the two constraining resources we would have to divide among the
subsystems. As we discovered through testing, the more pressing limit
came from the memory consumption of the software. We thus concluded
that we should execute the submodules sequentially, and divided up the
execution time among the different modules. At one end, we had the
Trip Generation, Distribution, Mode Choice, and the Greenhouse Gas
Calculator, which took very little time and required very little
memory to run. We didn't specify these programs technically because of
this.


The subsystem specifications are laid out in table \ref{specs}
\begin{table}[htp]
  \newlength\midcolumnwidth
  \midcolumnwidth=.74\textwidth plus 10\tabcolsep minus 10\tabcolsep
  \centering
  \caption{Specifications table}
  \label{specs}
  \begin{tabular}{%
    >{\raggedright}p{.11\textwidth}%
    p{\midcolumnwidth}%
    >{\raggedright\arraybackslash}p{.15\textwidth}}
  \firsthline
  \bfseries Module & \bfseries Qualitative Requirements & \bfseries
  Quantitative Performance Requirements \\ \hline
  Website & Aesthetically pleasing; easy to use & \\
  CGI Script& Modular and light. Interface with the summary statistic
  database generated by the back-end. & Memory footprint under 10 MB;
  & Startup time under 50 ms \\
  Back-end & Modular and testable & Runtime under one day; memory
  footprint under 3GB \\
  Trip Generation & Interface with a database of TAZs. Accept as input
  an expected attendance value.
  Output a list of trips specifying the origin of each one. &
  -- \\
  Trip Distribution & Assign arrival times to each trip taken from the
  Trip Generation module. & -- \\
  Mode Choice & Assign a mode choice to each entry in the the list
  from the trip distribution module.
  Output a vector of the different trips in each category. & -- \\
  SEPTA & Interface with the GTFS database to find optimal routes from
  all TAZs, subject to capacity constraints. Outputs summary
  statistics. & 1h for 10,000 users \\
  Cars & (See below) & 1h for 30,000 cars \\
  Cars - Micro & Accept the list of trips and read in a database of
  the road network in and around the stadium. Compute total distance
  and idling time per car to reach the highways/main roads. & \\
  Cars - Macro & Accept the same list of trips as above and retrieve
  the total highway and secondary road times across all trips from the
  car database. & \\
  GHG Calculator & Process the outputs from the trip assignment
  modules and compute an overall level of GHG emissions. & --  \\
  \lasthline
  \end{tabular}
\end{table}

\subsection{Hardware and Software Requirements}
\subsubsection{Hardware Requirements and Design Approach}
This project did not require special hardware. Part of the
specifications of the software were that it would run on a standard
server, and it did. Furthermore, the simplicity of the website ensures
that a modest server would be able to handle the small loads.

\subsubsection{ Software Requirements and Design Approach}
The software we created was written primarily in Python, with some C
code in the back-end and Javascript for the front-end. We also chose
to use SQLite for the database work. We chose Python for the
simplicity and power of the language. By designing our system in
separate parts for the front-end and the back-end, our software
achieved the most important requirements of modularity.

Using Python enabled us to use open-source tools like
NetworkX\cite{networkx} and igraph\cite{igraph}. For the performance
critical part of our back-end, we used C to write a Python extension
that would interface with GLPK\cite{glpk}. The main reason to use C
rather than Python was the memory savings involved in the change --
each Python object cost us an incremental 16 bytes. For the SEPTA flow
solver, which had tens millions of nodes and edges the incremental
memory consumption pushed the model near the limits of a standard
computer.

The requirements we have from our software will be as follows:
\label{requirements}
\begin{description}[style=nextline]
    \item[Modularity] We should we able to switch out one
  implementation of the train model, for instance, and replace it
  seamlessly with another, more efficient implementation.
    \item[Scalability] It should be able to handle a large number of
  inputs and not break under scale. It will have to deal with tens of
  thousands of fans inhabiting the model.
    \item[User-Friendliness] We want the final output, or GUI, to be
  extremely user-friendly and it should be operational without a
  manual.
\end{description}