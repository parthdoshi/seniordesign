
\subsubsection{Test}
The goal was to test every aspect of the system specifciation as
stated in section 3.2.

The python \texttt{time} module was primarily used to time the server
response time to ensure that we were within the defined
specifications.  A logfile was created and timers created around each
of the three components of the server response (start-up script, user
input processing and forming the response HTML). It is important to
note that the values were not written to the logfile until after the
processing had finished (since the overhead I/O cost would skew the
performance estimates).

The memory footprint of the CGI script was tested using the standard
library module \emph{resource}. The module provides a memory-checking
function called getrusage() that has an attribute ru_maxrss that gives
total memory usage for the calling process. We were able to test that
we met the specification set out for the CGI script (10 MB).

The aesthetic appeal and functionality were tested by demoing the
website to sample users who provided feedback on the user experience
and layout of the website. This was utilized to refine the design and
add features (such as an explanation of how the emissions calculator
works), in order to make the website more informative, easy-to-use and
functional.

The website was also required to be able to handle errors
gracefully. Every input box on the website was error-checked using
JavaScript regular expressions to dynamically verify that the user had
entered a valid input. If not, the user would be displayed a pop-up
window that suggested they try another input. This was verified
manually for the range of inputs that were deemed ``invalid''.

As for the backend, the testing was conducted on sub-networks of the
larger SEPTA network to ensure that the nodes and trips were computed
correctly. Once that was verified, the models were run on the
full-scale system in order to ensure that the execution completed
within a day (as determined by the frequency of baseball games). In
addition, the models were run under memory-checking tools, including
\textt{Valgrind} (for the programming language C), in order to ensure that the
model did not leak memory or utilize more than was specified in the
system specification.

\subsubsection{Demonstration}

On demo day we had two deliverables: the project poster and the
mock-up of the web application.

The poster aimed to provide a quick, concise overview of our
project. It contained the abstract, a definition of the problem we are
trying to solve and our project objectives. It also contained diagrams
of the car network that we created, the backend and frontend system
block diagrams as well as a sample block diagram for a specific
subsystem (the SEPTA subsystem). We also discussed validation
techniques that were utilized and presented some graphical results
from our model to highlight the benefits of switching from cars to
SEPTA.

The website was a basic web application that is described in detail in
the Results section below. It is aimed at informing users about the
tradeoff of choosing to take a car to the game in place of public
transportation as well as providing information about the next home
game and the weather at the time of the next home game. Additionally,
alternative measures to reduce carbon impact are suggested for users
who were averse to the switch. The website was created to be
interactive, where judges and other observers were able to select from
a limited set of zip codes to see how the user would interact with the
finished product.
