The schedule consisted of the different tasks that we estimated we would
need to accomplish in order to have a successful project. The tasks
can be broken into

\begin{itemize}
  \item Planning and design
  \item Creating the required subsystems
  \item Integrating the subsystems into one system
  \item Testing and Validation.
\end{itemize}

Each task was assigned to only one individual although others may also have
been working on the task, but the assigned individual was accountable to
get it done correctly and on time. Based on the Gantt chart and
schedule it is clear that we aimed to work consistently over the course
of the year, including winter break, to meet the milestones (course requirements) and not fall behind schedule. Although there were delays in completing some tasks, we did not let that disrupt our uniform workflow by reallocating resources to other tasks and either getting them started early or completed faster by having more people working on it. 

The main challenge that we faced was the delay and inability to establish contact with the Philadelphia Phillies or Philadelphia Police Department. The lack of contact had lead us to postpone some of our earlier tasks till we met them to understand their requirements and get the required data from them. Our advisor Dr. Huemmler worked very hard to try to get us in contact with them and sent multiple emails and even letters on official University of Pennsylvania SEAS letterhead to five people at the Phillies and Police department, but still did not manage to schedule a meeting.

Finally, during mid March, we reached our deadline to determine the end users and had to disregard the Phillies and make the sports fans the end users. This lead two major consequences for the schedule: 

\begin{enumerate}
  \item Changing some of the originally scheduled tasks
  \item Non-uniform distribution of work
\end{enumerate}

Since the end users changed, there were certain aspects of the system that needed to be removed and other subsystems and functionalities that had to be added to accommodate the change in requirements. The Incentive Scheme or Tailgate Model had to be removed because now we no longer had to determine the effect of various incentive schemes on trip distribution or modal distribution. However, we had to develop a website for the front end that synthesized the information from the simulation and GHG (Green House Gas) Emissions model. We also had to increase the scope of the SEPTA model to incorporate the entire SEPTA system rather than just the station level model for AT&T Station. Finally, we had to incorporate travel time into both the SEPTA and Car models because that would be an important output for the end user.

Due to this major change when there was approximately one month remaining before demo day, there was a lot of work that needed to be done and we had to allocate more time to the project that we had originally scheduled. However, we were able to do this for two main reasons. Firstly, we had anticipated this and as accordingly set a contingency plan, which involved pushing some tasks back that were the same for both systems, and pushing other tasks forward that were specific to either one of the systems. Secondly, when we were originally making the schedule we had kept slack time in the tasks. There was a third unanticipated factor of over-estimating the time it would take to complete certain tasks, which helped us at the end. We had over-estimated the time it would take to make the SEPTA subsystem and GHG Emissions model, however, the time we had saved in making the model originally, we spent on updating it to suit the new requirements. However, developing the new front-end system took a lot of time and lead to greater than proportionate time spent on the project in April.

Despite the late scheduling changes, we managed to have a working model and front-end website by demo day. However, if we were to do this again we would have set the deadline to determine the end user to a much earlier date, such as the end of first semester. This would have given us more time to refine the front-end and even launch the website before demo day in order to get user feedback and incorporate that into our model. However, we used alternative methods of testing and validation.

