Our project was able to establish concrete evidence to back a switch
from cars to public transportation as a way to reduce carbon emissions
from fans traveling to/from sports games.

Our model generated some very interesting conclusions:
\begin{itemize}
    \item 20\% of fans would save time by using SEPTA instead of
  driving to games due to the time spent idling at the stadium after a
  game.
    \item \~40000 fans face a 20 minute tradeoff between SEPTA and
  driving when accounting for idling time at the stadium when trying
  to get out after a game.
    \item If 12,000 fans switched to SEPTA, we would have emission
  savings of 20 million grams of carbon dioxide one-way.
    \item If 40,000 fans switched to SEPTA, we would have emission
  savings of 100 million grams of carbon dioxide one-way.
\end{itemize}

In addition to carbon savings, fans would also benefit from added
convenience of not having to drive to the game while making a more
environmentally-conscious choice about how they support their favorite
sports team.

In terms of future work, there are a few opportunities that can be
expanded upon:
\begin{itemize}
    \item Mobile Applications for iOS/Android/Windows Phone utilizing
  the algorithms and models we have established.
    \item Refine the algorithms to determine optimal choice of gate
  openings to route traffic out of stadiums quicker.
    \item Extend the findings to other sports teams within the South
  Philadelphia Sports Complex (Eagles, Flyers, 76ers).
    \item Extend the findings and algorithms to other sports teams
  across the country.
    \item The algorithms and models in place are structured to be
  adaptable and would be easy to modify for new teams.
\end{itemize}
