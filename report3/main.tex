\documentclass[12pt]{article}
\usepackage{sd}

\title{Stadium Traffic}
\team{13}
\addauthor{Parth Doshi;parthd@seas.upenn.edu}
\addauthor{Zain Mukaty;zmukaty@seas.upenn.edu}
\addauthor{Felipe Ochoa;felipeo@seas.upenn.edu}
\advisor{Dr. Andrew E. Huemmler;huemmler@seas.upenn.edu}
\reportname{Phase 2 Project Report}
\addbibresource{sources.bib}

\begin{document}
\maketitle
\abstract


Citizens Bank Park, the home stadium of the Philadelphia Phillies, is
located within city limits and is accessible by car, train and bus. In
the 81 home games of their 2012 season, the Phillies had an average
attendance of 44,021, for a total of over 3.5 million fan trips. All
these fans generate greenhouse gases traveling to and from the
stadium.

The Phillies have started the ``Red Goes Green'' initiative to reduce
their environmental impact by installing solar panels, planting trees,
and buying offsets for their carbon emmissions. They also offer tips
for fans who want to help with this mission.  No significant efforts
have been made so far, however, to quantify and reduce the
environmental footprint of these fans traveling to and from the
stadium.

This system meets two important needs that would help the Phillies and
their fans further reduce their carbon footprint.

First, it estimates the total amount of emissions currently generated
in based on the average distance traveled by fans and on the types of
transportation they use. These parameters are in turn estimated by a
comprehensive traffic model.

Second, the system provides fans information regarding alternative
travel options to and from the stadium by interfacing with the SEPTA
website. The tool is sensitive to the unusually-high demand placed on
transportation infrastructure around the game, and it gives fans
information on the congestion levels, travel time, and environmental
impact of each travel option.


\newpage

\tableofcontents

\newpage

\mainmatter

\section{Introduction}

\subsection{Overview}
Stadium Traffic aims to reduce the greenhouse gas (GHG) emissions from
people going to and from sports games. In particular, our project will
focus on the Philadelphia Eagles and Phillies stadiums.  We aim to
build a model that can be used to quantify the GHG emissions generated
due to people coming to and going from games. We will then develop
solutions to reduce the number of people coming by cars or leave over
longer time period and develop a traffic routing algorithm, aiming to
reduce the GHG emissions from idling cars waiting to leave the
stadium. We will test these on our model to find a successful
combination. If we can get the cooperation of Phillies/Eagles/Flyers
and Philadelphia Police Department, we hope to work with them to
implement the results of our project.

\subsection{Motivation}
All of us have been to large gatherings like sports games, concerts or
conferences and can therefore relate to the traffic problems when
everyone is trying to leave at the same time at the end of the
event. However, what most people don't realize is the environmental
effect of this traffic. Our project aims to find solutions to reduce
this traffic and hence reduce the emissions while cars idle in parking
lots trying to leave.

The project is focusing on the Philadelphia Eagles, Phillies and
Flyers stadiums. Sports events draw huge crowds at peak traffic hours
and are therefore a great instance on which to model our project. The
Eagles are also undertaking a Go Green! initiative, which is a push
towards becoming more environmentally conscious. Some initiatives that
they have taken are the use of renewable energy sources for their
power consumption and using cups and plates made from recycled
materials. We believe that we can get the support of the Eagles for
our project, as it can be part of the Go Green! initiative and further
reduce the carbon footprint of the stadium.

When the football or baseball game ends, everyone tries to the leave
the stadium at the same time. Although traffic police conduct the
traffic, they do it in a haphazard way. Furthermore, the decision of
which gates to be open and which ones to remain closed is also done
using intuition rather than any efficiency-maximizing algorithm. We
saw this inefficient process as an opportunity to develop an algorithm
to reduce car idling time and GHG emissions.

\subsection{Project Goal and Objectives}
Although the overarching aim of our project is to reduce GHG
emissions due to people going to/from the sports games, we broke this
down into certain specific and measurable objectives:

\begin{enumerate}
    \item Develop a comprehensive model of transportation to and from
  the stadium.
    \item Quantify the total emissions due to transportation to/from
  games.
    \item Develop a tool to simulate the emissions impact of various
  initiatives to reduce private transportation usage (e.g. offering a
  discounted drink to someone who rides the SEPTA).
    \item Create an algorithm to optimally route exiting vehicle traffic
  according to total emissions generated.
    \item Develop a method for the Philadelphia traffic police to
  implement the traffic routing recommendations of our project.
\end{enumerate}

The success of each of the objectives can be measured and how we would
measure the success is stated below:

We would consider a success model one that takes inputs such as trip
distribution, number of attendees, start and end time and produces
outputs such as average idling time and number of cars.

The aim of the project is to reduce GHG emissions, so we need to be
able to measure and quantify GHG emissions from idling cars.

Our model should be able to quantify the effect of different
incentives on GHG emissions so that we can test different
strategies and find the optimal combination.

Optimal routing of traffic can impact GHG emissions from idling
cars, so we need to create an algorithm that can be constraint
optimized. A working algorithm can help the traffic police conduct
traffic in a more organized manner.

Having a front-end system that can be used by the Philadelphia Police
will allow our algorithm to be implemented and hence allow our project
to have an impact. Therefore we need a mock-up of a computer or phone application
that can be used by the police while conducting traffic.

\subsection{Constraints}
The main constraint is the cooperation of the Philadelphia
Eagles/Phillies/Flyers and Philadelphia Police Department because without
their cooperation we cannot implement our project. However, our
advisor Dr. Huemmler has good relationships with both sets of organizations
and has been trying to get their support for the project.

Another constraint is the team's lack of experience with
transportation engineering.  Since the project involves the
understanding, modeling and optimizing of transport systems, this
would be a constraint. However, we have taken an initiative to get the
help of Professor Vukan Vuchic, a senior professor in transport
engineering, and began familiarizing ourselves with relevant transport
systems concepts under his guidance.

Finally, our last constraint is the data collection required to build
a representative model. However, we have been trying to get the
required information from the Eagles, Phillies, Flyers and other government
departments.



\section{Discussion of Previous Work}


\subsection{Scholarly Work}
Scholarly research in transportation model distinguishes between two
broad approaches: \cite{kitamura1988}

\begin{description}[style=nextline]
    \item[Activity-based] Modeled at the individual level. Considers
  trips arising from different activities that comprise a tour. (E.g.,
  travel to kids' school, then to work, then to the grocery store, and
  finally back home). \cite{kitamura1988}
    \item[Trip-based] Modeled at the Traffic Analysis Zone (TAZ)
  level. Considers broader characteristics such as demographics of a
  neighborhood to model trip times and volumes.\cite{murthy01}
\end{description}

The activity-based model dominates in the most recent academic
literature, but from our survey, the trip-based model still appears to
be the most frequently used type of model by government
agencies. Within the trip-based models the dominant kind appears to be
the four-stage model, which breaks down into four components:
\cite{murthy01}

\begin{enumerate}
    \item Trip Generation
    \item Trip Distribution
    \item Mode Choice
    \item Trip Assignment
\end{enumerate}

In all the applications we've seen, each of those stages is further
refined by the category of the trip being modeled. The usual breakdown
is:

\begin{description}
  \item[Home-based work] (HBW) Trips from home to work or vice versa.
  \item[Home-based shop] (HBS) Trips from home to go shopping and back
(sometimes omitted).
  \item[Home-based other](HBO) Other trips having the home as an
origin or destination.
  \item[Non-home-based work] (NHW) Trips to or from work not in HBWl
  \item[Non-home-based other](NHO) All other kinds of trips.
\end{description}

One of the main benefits of the trip-based approach is its long
history and current widespread use. The model is also readily
modularized, and can be easily adapted and specialized for our
purposes. On the other hand, it is more simplistic than the
activity-based model, since it ignores the interactions of past trips
generated on future trip generation. The activity-based model,
however, has a clear computational disadvantage. The level of detail
the model reaches means that it is very demanding computationally, and
is also hard to break into subcomponents.

Our approach will combine aspects from both styles (as detailed in
section 3), but will borrow the terminology of the trip-based style.

\subsubsection{Phoenix, Arizona Planning Authority}
In 2011 the planning authority in Phoenix, Arizona, MAG, undertook an
extensive project to model transit movement to planned special events
in the region. \cite{kuppam11} Our chosen topic, sports games, falls
under their category of a planned special event. The authors identify
the proportion of special events patrons who utilized light rail
v. alternative modes of transportation and their approach to modeling
demand and modal choice for this event. While the results are specific
to their application, the methodology can be adapted to our specific
focus.

\subsubsection{ITE Trip Generation Database}
The ITE compiled a database with various data patterns related to Trip
Generation. \cite{ite08} We will be able to use the information it contains,
specifically the information related to the special events travel,
e.g. sports games and movies.

\subsubsection{Robertson Stadium}
Gunda Corporation performed a relatively similar project to the one we
are developing for Robertson Stadium at the University of
Houston.\cite{gunda} Although the stadium has a lower capacity than
the Eagles stadium, \cite{robertson-stadium} it is home to an MLS
team, and has to abide by all the traffic management principles of a
professional sports team. The scope and aim of Gunda's analysis
differed from our own, but their analysis can strongly inform a
component of our model. (The micro-car module, see Section \ref{cars})

\subsection{Group Members' Prior Work}
One of our group members has had experience working with Penn Transit
to improve the dispatching efficiency of their bus operations around
campus. This work has allowed for in-depth understanding of queuing
theory and demand generation, which will be extremely useful in our
module about fans taking cars to the game. Using elements of the prior
project, combined with additional concepts from traffic engineering,
we will be able to develop an accurate model.



\section{Strategic Plan/Structure}



\subsection{System Proposed Approach}

\subsection{System Specification}

We derive the specifications for our system based on the requirements
set by their running environments, especially those of the GUI. Since
we chose a website for this interface, the primary requirement was for
there to be a response time of under two seconds, since this is the
industry-standard benchmark \cite{akamai}, and what sites such as
Google use \cite{two-seconds}.

Response time for our website is driven by a variety of factors
typically broken down into:

\begin{itemize}
  \item The number of \emph{resources}
  \item The DNS lookup time
  \item The server response time
  \item The download time
  \item The rendering time
\end{itemize}

These times can be seen in a benchmark of an early version of our
website in figure \ref{fig-website-benchmark}.

\begin{figure}[htp]
  \centering
  \includegraphics[height=8cm]{graphics/performance.png}
  \caption{A `Waterfall View' of the stages of a website retrieval. In
  dark green \emph{DNS Lookup}; in orange \emph{Initial Connection};
  in light green \emph{Server processing time}; in blue \emph{Download
  Time}. The green bar shows the beginning of the rendering process,
and the blue line shows the end.}
  \label{fig-website-benchmark}
\end{figure}

The main parameters under our control for this interface were the
\emph{number of resources} and \emph{download time} and the \emph{server
  response time}. Based on some experimentation, we developed the
benchmarks for the first two items shown in table \ref{specs}. The
benchmark for \emph{server response time}, as discussed above, was the
main constraint in the design of this front end, so we set the
specification for this part as high as possible while remaining
withing the overall 2 second requirement, as shown in table
\ref{specs}.

To generate the response, the server must carry out 3 tasks:

\begin{itemize}
    \item Start-up script
    \item Process user input
    \item Form response HTML
\end{itemize}

By timing a variety of CGI scripts, and comparing the results to
published benchmarks\cite{cgi-benchmark} we determined an acceptable
performance range for the start-up time. The remaining time from the
original 2 seconds was then allocated jointly to processing user input
and forming the response HTML. The requirements are set out in table
\ref{specs}.

Given the design for our overall system, [FINISH ME]


The program must handle improper input robustly and must be able to
deal with exceptions and errors gracefully. Specifically, a user of
average computer literacy should be able to understand error messages,
and, in a worst-case scenario, should be able to simply reset the
application to restore functionality.

The subsystem specifications are laid out in table \ref{specs}
\begin{table}[htp]
  \newlength\midcolumnwidth
  \midcolumnwidth=.74\textwidth plus 10\tabcolsep minus 10\tabcolsep
  \centering
  \caption{Specifications table}
  \label{specs}
  \begin{tabular}{%
    >{\raggedright}p{.11\textwidth}%
    p{\midcolumnwidth}%
    >{\raggedright\arraybackslash}p{.15\textwidth}}
  \firsthline
  \bfseries Module & \bfseries Qualitative Requirements & \bfseries
  Quantitative Performance Requirements \\ \hline
  CGI Script & Modular and correct & Memory footprint under 10 MB;
  Startup time under 50 ms \\
  \lasthline
  \end{tabular}
\end{table}
\subsection{Hardware and Software Requirements}
\subsubsection{Hardware Requirements and Design Approach}
This project did not require special hardware. Part of the
specifications of the software were that it would run on a standard
server, and it did. Furthermore, the simplicity of the website ensures
that a modest server would be able to handle the small loads.

\subsubsection{ Software Requirements and Design Approach}
The software we created was written primarily in Python, with some C
code in the back-end and Javascript for the front-end. We also chose
to use SQLite for the database work. We chose Python for the
simplicity and power of the language. By designing our system in
separate parts for the front-end and the back-end, our software
achieved the most important requirements of modularity.

Using Python enabled us to use open-source tools like
NetworkX\cite{networkx} and igraph\cite{igraph}. For the performance
critical part of our back-end, we used C to write a Python extension
that would interface with GLPK\cite{glpk}. The main reason to use C
rather than Python was the memory savings involved in the change --
each Python object cost us an incremental 16 bytes. For the SEPTA flow
solver, which had tens millions of nodes and edges the incremental
memory consumption pushed the model near the limits of a standard
computer.

The requirements we have from our software will be as follows:
\label{requirements}
\begin{description}[style=nextline]
    \item[Modularity] We should we able to switch out one
  implementation of the train model, for instance, and replace it
  seamlessly with another, more efficient implementation.
    \item[Scalability] It should be able to handle a large number of
  inputs and not break under scale. It will have to deal with tens of
  thousands of fans inhabiting the model.
    \item[User-Friendliness] We want the final output, or GUI, to be
  extremely user-friendly and it should be operational without a
  manual.
\end{description}

\addtocounter{subsection}{1} %\subsection{Test and Demonstration}
%\subsubsection{Test}
%\subsubsection{Demonstration}

\subsection{Schedule}
The schedule consists of the different tasks that we estimate we will
need to accomplish in order to have a successful project. The tasks
can be broken into

\begin{itemize}
  \item Planning and design
  \item Creating the required subsystems
  \item Integrating the subsystems into one system
  \item Implementation.
\end{itemize}

Each task is assigned to only one individual although others may also
be working on the tasks, but the assigned individual is accountable to
get it done correctly and on time. Based on the Gantt chart and
schedule it is clear that we aim to work consistently over the course
of the year to meet the milestones (course requirements) and also aim
to get some work done over winter break so that we don’t fall behind
schedule.

We are currently ahead of schedule in some tasks, and behind schedule
in other tasks. This is similar to our situation when we submitted the
first project report because we are facing similar obstacles in
obtaining required information to progress in certain areas. However,
this is not as much of a concern to us as being completely behind
schedule because we can alter resource allocation to reconcile the
variance and bring the project back on schedule.

The two areas where we are behind schedule are:

\begin{enumerate}[label=(\emph{\alph*})]
  \item Working with the Philles/Eagles
  \item Working with the Philadelphia Police Department.
\end{enumerate}

This can be attributed to the major constraint mentioned in the
introduction. Dr. Huemmler has been trying to get us in touch with
them, but so far they have not been very responsive.

On November 16th, 2012, Dr. Huemmler sent letters on SEAS letterhead
paper to the following five individuals in these organizations:
\begin{description}[style=nextline]
    \item[Mike DiMuzio]
  Director of Operations/Facilities, Philadelphia Phillies
    \item[Julie Hershey]
  Community Relations, Philadelphia Eagles
    \item[Zach Hill]
  Senior Director of Communications, Philadelphia Flyers
    \item[Michael Preston]
  Director of Public Relations, Philadlephia 76ers
    \item[Lt. John Hewitt]
  PSA Lieutenant
\end{description}

Dr. Huemmler assures us that we will have meetings set up for the
middle of January, when we return from winter break. We have formed a
contingency plan, and updated the schedule so that we can allocate
more time to this to bring us back on schedule once we have
established contact. We plan to devote more resources to the subtasks
that rely on the meetings in order to bring the project back on
schedule.

Although the delay in contract with the Eagles and Philadelphia PD
lead us to fall behind in certain areas, we were able to reallocate
the time and resources that should have been used there, to other
tasks. These tasks include the SEPTA subsystem and GHG emissions
subsystem. Although we had originally planned to start these later in
this semester and next semester (respectively), we were able to start
and make progress on both of these subsytems during this semester.

Another factor that helped us get ahead of schedule in these areas was
the fact that we over-estimated the time to complete specific
tasks. The specific tasks that we over-estimated were the research to
find the necessary data for the GHG emissions subsystem and the
programming aspect of the SEPTA subsystem.

Overall, at the end of the first semester we have made significant
progress, which is discussed in more detail in the Results section. We
have defined our problem and scope, designed our systems approach and
began building two of the subsystems. We also plan to have some work
done over winter break, and consolidate our efforts at the beginning
of next semester and revise our schedule accordingly.

\section{Preliminary Results}
As discussed in the schedule discussion in section 3.5, we have made
significant progress on our project that will be critical to
supporting our efforts in Phase 2. We have had four main outcomes from
Phase 1 that we hope to carry forward.

\subsection{Project scope and determination of target user}

Our project started off with the aim of reducing CO2 emissions from
cars idling in parking lots after sports games. Over the course of
Phase 1, we have broadened our scope to develop a system to reduce GHG
emissions as a byproduct of fans travelling to and from sports
games. This broad aim will be carried forward to Phase 2 to drive us
to find innovative solutions to problems that we face by keeping the
overall goal in mind.

Through the semester, we have also determined three possible target
users for the finished product from our project:

\begin{descenum}
    \item[Philadelphia Sports Teams] (Eagles/Phillies/Flyers) We could
develop an interface that would allow the management of these teams to
determine the impact of various incentives towards reducing GHG
emissions.

  \item[Philadelphia Police Department] We could develop a mock-up of
an application that would help streamline traffic direction for fans
leaving stadiums

  \item[Sports Fans] We could develop a web interface/mock-up that would
allow users to understand the GHG emission impact of their travels and
alternatives to reduce this impact.
\end{descenum}

Our work in Phase 2 will involve determining which target user would
be the best choice to focus on given the availability of data and
involvement of the sports teams (as talked about before in the
report).

\subsection{Overall systems approach}
As discussed in section 3.1, we
have developed an overall systems approach to the project that we will
be using to guide our progress next semester. The system block diagram
will be used to understand how each module fits into the overall
system and will help prioritize among competing choices for time
allocation.

\subsection{SEPTA Subsystem}
Through Phase 1, we have developed a comprehensive database of all
SEPTA rail, trolley and transit stations and the lines that operate
across these stations. This data is stored in a well-typed XML based
on a DOM that ensures that all required fields are present for each
station. This database will be used to support further development of
the train model in calculating the emissions from the train system as
well as travel time alternatives when suggesting substitution of car
travel with train travel.

\subsection{GHG Emissions Subsystem}
Through Phase 1, we
have developed a comprehensive plan to create the GHG emissions
subsystem. The discussion above already detailed the creation and
design of the system, including the input, transfer function and
output for each of the five iterations. This plan will be used in
Phase 2 to program the subsystem and create a working model of the GHG
emissions subsystem.

\addtocounter{section}{1} % \section{Lessons Learned}

\section{Equipment/Fabrication/Software Needs}
We do not have any need for specialized equipment or software. The
software we will be using is either publicly available, or available
at Penn. We expect that our production version of the model will not
depend on commercial tools.

\addtocounter{section}{1} % \section{Conclusions and Recommendations}

\section{Nomenclature}
{\centering
\midcolumnwidth=.8\textwidth plus 10\tabcolsep minus 10\tabcolsep
\begin{tabular}{%
    >{\raggedright\bfseries}p{.1\textwidth}%
    p{\midcolumnwidth}}
  API & Application programming interface. An official, documented way
  for a program to access a database or another program, such as that
  of Google Maps. \\
  DOM & Document Object Model. A file format that is very easy to
  parse and yet human-readable. \\
  EPP & Emissions per passenger. \\
  EPM & Emissions per mile. \\
  GHG & Greenhouse gases. We mostly mean Carbon Dioxide, but we use
  the term GHG since emissions are highly correlated across types in
  this application. \\
  HBO & Home-based other. A category of trips in the
  standard  model. Contrast with HBS, HBW, NHO, and NHW. \\
  HBS & Home-based shop. A category of trips in the standard
  model. Refers to trips from the home to go shopping. Contrast with
  HBO, HBW, NHO, and NHW. \\
  HBW & Home-based work. A category of trips in the standard
  model. Refers to trips from the home to go to work. Contrast with
  HBO, HBS, NHO, and NHW. \\
  MAG & Maricopa Association of Governments. It is in charge of
  transportation planning for Phoenix and it surroundings. \\
  NHO & Non-home-based other. A category of trips in the standard
  model. Contrast with  HBO, HBS, HBW, and NHW. \\
  NHW & Non-home-based work. A category of trips in the standard
  model. Refers to trips not from the home to go to work. Contrast
  with HBO, HBS, HBW, and NHO. \\
  SEPTA & Southeast Pennsylvania Transit Authority. Manages the
  commuter rail, the subway, trolley, and bus systems in and around
  Philadelphia \\
  SQL & A type of database that is commonly used. \\
  TAZ & Traffic analysis zone. Used in modeling the UTMS. It involves
  dividing a geographical area into units that have sufficiently
  similar transit and demographics to treat as one for the purposes of
  the model. \\
  UTMS & Urban Transportation Modeling System. A commonly used
  framework to model traffic demand and flows. \\
  XML & Extensible markup language. A structured file format that is
  human-readable and easily parseable and customizable. \\
\end{tabular}
}

\makereferences

\makebibliography


\section{Financial Information}
We expect our project to incur minimal cost. At this time, the only
foreseeable expenses are the cost of transportation to the
stadium. Depending on how we refine the user needs after our meeting
with the Eagles, we may have to purchase a mobile device on which to
prototype our model interface.

%\section{Ethical Issues}

\appendix

\section{A Documented Module of Code}
\lstinputlisting[language=python,basicstyle=\small]{includes/module.py}
\end{document}

%  LocalWords:  DOM EPP EPM GHG Huemmler Vukan Vuchic ITE
