\documentclass[12pt]{article}
\usepackage{sd}

\title{Stadium Traffic}
\team{13}
\addauthor{Parth Doshi;parthd@seas.upenn.edu}
\addauthor{Zain Mukaty;zmukaty@seas.upenn.edu}
\addauthor{Felipe Ochoa;felipeo@seas.upenn.edu}
\advisor{Dr. Andrew E. Huemmler;huemmler@seas.upenn.edu}
\reportname{Phase 2 Project Report}
\addbibresource{../sources.bib}

\begin{document}
\maketitle
\abstract

Citizens Bank Park, the home stadium of the Philadelphia Phillies,
is located within city limits and is accessible by car, train and bus. In
the 81 home games of their 2012 season, the Phillies had an average
attendance of 44,021, for a total of over 3.5 million fan trips. All 
these fans generate greenhouse gases traveling to and from the stadium.

The Phillies have started the "Red Goes Green" initiative to reduce their 
environmental impact by installing solar panels, planting trees, and
buying offsets for their carbon emmissions. They also offer tips for fans who want to help with this mission.
No significant efforts have been made so far, however, to quantify and 
reduce the environmental footprint of these fans traveling to and from 
the stadium.

This system meets two important needs that would help the Phillies and
their fans further reduce their carbon footprint.

First, it estimates the total amount of emissions currently generated in 
based on the average distance traveled by fans and on the types of 
transportation they use. These parameters are in turn estimated by a 
comprehensive traffic model.

Second, the system provides fans information regarding alternative travel
options to and from the stadium by interfacing with the SEPTA website.
The tool is sensitive to the unusually-high demand placed on transportation infrastructure around the game, and it gives fans information on the congestion levels, travel time, and environmental impact of each travel option.

\end{document}
